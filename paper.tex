\documentclass[11pt]{jres}

\title{Titre de l'article JRES 2013}

\auteur{Prénom Nom de l'auteur 1}{Laboratoire ou service...}{%
1, rue de l'adresse\\
34\,000 Montpellier}

\auteur{Prénom Nom de l'auteur 2}{Laboratoire ou service}{%
2, rue du second auteur\\
34\,000 Montpellier}

\date{}


\begin{document}
\maketitle

\begin{abstract}
Ce document résume les instructions aux auteurs de la conférence JRES 2013.
Il est téléchargeable sur le site des JRES : \url{http://www.jres.org}.

Le résumé doit faire de quinze à vingt lignes. Il est écrit en italique, 
Arial Narrow ou Liberation Sans Narrow, 10 points. 
Le titre « Résumé » (ainsi que le titre « Mots-clefs » ci-dessous) 
est en 12 points, gras.
\end{abstract}

\motscles{IPv6, PKI, CMS …}

\section{Introduction}

Voici les instructions de présentation pour la rédaction d'un article
pour les JRES.  

Il est recommandé d'utiliser prioritairement le
traitement de texte OpenOffice.org Writer ou LibreOffice, et il est
impératif d'utiliser le format Open Document Texte (.odt) pour
soumettre un article au comité de programme. 
La version d'OpenOffice doit être au moins 3.2.1.  

Les auteurs doivent utiliser les modèles téléchargeables sur le site
des JRES (\url{https://2011.jres.org/auteurs/modeles}).

\section{Présentation générale}

\subsection{Structuration et polices}

La police de base est de type \emph{Arial Narrow}

Le texte  normal est en 10 points.

La police pour les extraits de fichiers de configuration ou de code source est \texttt{Courier New} en 8 points.

L'environnement pour les extraits de fichiers de configuration ou 
de code source  est « lstlisting ». Voici un exemple :

\begin{lstlisting}
10.0.0.1   serveur-1
127.0.0.1  localhost.localdomain localhost
\end{lstlisting}

\end{document}



