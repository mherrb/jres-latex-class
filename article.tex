%% Copyright 2012 CNRS LAAS 
%
% This work may be distributed and/or modified under the
% conditions of the LaTeX Project Public License, either version 1.3
% of this license or (at your option) any later version.
% The latest version of this license is in
%   http://www.latex-project.org/lppl.txt
% and version 1.3 or later is part of all distributions of LaTeX
% version 2005/12/01 or later.
%
% This work has the LPPL maintenance status `maintained'.
% 
% The Current Maintainer of this work is Matthieu Herrb
%
% This work consists of the files:
% jres.cls
% article.tex

\documentclass[11pt]{jres}

\title{Titre de l'article JRES 2013}

\auteur{Prénom Nom de l'auteur 1}{Laboratoire ou service...}{%
1, rue de l'adresse\\
34\,000 Montpellier}

\auteur{Prénom Nom de l'auteur 2}{Laboratoire ou service}{%
2, rue du second auteur\\
34\,000 Montpellier}

\date{15/10/2013}

\begin{document}
\maketitle

\begin{abstract}
Ce document résume les instructions aux auteurs de la conférence JRES 2013.
Il est téléchargeable sur le site des JRES : \url{http://www.jres.org}.

Le résumé doit faire de quinze à vingt lignes. Il est écrit en italique, 
Arial Narrow ou Liberation Sans Narrow, 10 points. 
Le titre « Résumé » (ainsi que le titre « Mots-clefs » ci-dessous) 
est en 12 points, gras.
\end{abstract}

\motscles{IPv6, PKI, CMS …}

\section{Introduction}

Voici les instructions de présentation pour la rédaction d'un article
pour les JRES.  

Il est recommandé d'utiliser prioritairement le
traitement de texte OpenOffice.org Writer ou LibreOffice, et il est
impératif d'utiliser le format Open Document Texte (.odt) pour
soumettre un article au comité de programme. 
La version d'OpenOffice doit être au moins 3.2.1.  

Les auteurs doivent utiliser les modèles téléchargeables sur le site
des JRES (\url{https://2011.jres.org/auteurs/modeles}).

\section{Présentation générale}

\subsection{Structuration et polices}

La police de base est de type \emph{Arial Narrow}

Le texte  normal est en 10 points.

La police pour les extraits de fichiers de configuration ou de code source est \texttt{Courier New} en 8 points.

L'environnement pour les extraits de fichiers de configuration ou 
de code source  est « lstlisting ». Voici un exemple :

\begin{lstlisting}
10.0.0.1   serveur-1
127.0.0.1  localhost.localdomain localhost
\end{lstlisting}

Les titres de section sont en 14 points gras, avec numérotation arabe.

Les titres de sous-section sont en 12 points gras, avec numérotation arabe (avec rappel du numéro de section et séparation par un point).

Les titres de niveau inférieurs sont à éviter, pour des raisons de lisibilité.

Les notes de bas de page sont en 8 points\footnote{Comme ceci}. 

\subsection{Typographie}

Les auteurs s'attacheront, dans la mesure du possible, à respecter les
règles typographiques françaises.

En particulier, ils veilleront à utiliser des espaces insécables avant
les ponctuations hautes (;:!?) et des guillemets français (« »).

\subsection{Bibliographie}

La bibliographie sera faite avec des numéros et dans l'ordre d'apparition dans le texte.

\begin{enumerate}

\item Pour les articles d'une revue, on indiquera obligatoirement les
  auteurs, le titre de l'article, la revue, le numéro, le volume,
  l'année et le mois. Voir \cite{exemple1}

\item Pour les articles dans les actes d'un congrès, on mentionnera
  obligatoirement les auteurs, le titre de l'article, le titre des
  actes, l'année. Voir : \cite{exemple2} (le mois et le lieu sont
  facultatifs mais conseillés)

\item Pour les livres, on citera obligatoirement le ou les auteurs, le
  titre, l'éditeur et l'année. Voir : \cite{exemple3} (il est conseillé de
  rajouter le lieu d'édition)

\end{enumerate}

\subsection{Adresses électroniques}

Pour diminuer les risques de moissonnage, aucune adresse mail ne
figurera dans l'article.

\section{Mise en page}

\subsection{Nombre de pages}

Le nombre de pages de l'article relatif à un poster doit être compris
entre une et sept pages. Le poster lui même sera inséré avec cet
article. Pour les démonstrations et les présentations courtes,
l'article doit comprendre entre deux et huit pages. Pour les autres
présentations, le nombre de pages d'un article doit être compris entre
six et huit pages.

\subsection{Marges}

Les marges sont les suivantes :

\begin{itemize}
\item en haut : 2,5\ cm ;
\item à gauche et à droite : 2\ cm ;
\item en bas : 3\ cm.
\end{itemize}

Les pages doivent être numérotées. Le pied de page du modèle doit être
maintenu.

\subsection{Figures}

Les figures sont de préférence insérées à proximité du texte qui y
fait référence. Elles sont numérotées et suivies d'une légende
explicite, en italique, non gras. Le texte fait référence à la 
Figure~\ref{fig-exemple}.

\begin{figure}
\centerline{\includegraphics[width=7cm]{figure}}
\caption{Un exemple de figure}
\label{fig-exemple}
\end{figure}


\section*{Annexe}

Les auteurs ajoutent, si nécessaire, une annexe avec une section non numérotée.

% Bibiliographie
\nocite{*}
\bibliography{article}

\end{document}
