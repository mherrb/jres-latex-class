%% Copyright 2012 Renater
%
% This work may be distributed and/or modified under the
% conditions of the LaTeX Project Public License, either version 1.3
% of this license or (at your option) any later version.
% The latest version of this license is in
%   http://www.latex-project.org/lppl.txt
% and version 1.3 or later is part of all distributions of LaTeX
% version 2005/12/01 or later.
%
% This work has the LPPL maintenance status `maintained'.
%
% The Current Maintainer of this work is Matthieu Herrb
%
% This work consists of the files:
% jres.cls
% article.tex
%----------------------------------------------------------------------
%
% Instructions pour utiliser la classe jres :
%
% Suivre cet exemple au maximum. Règles générales :
%
% - suivre les règles de saise de french/babel concernant les
%   ponctuations doubles
% - le style de bibliographie est imposé, pas besoin de le re-définir.
%
%----------------------------------------------------------------------
% Utilisation de la classe de documents jres.
% Les parametres city et year permettent de situer le lieu et l'année
% (optionnels si la version à jour de la classe est utilisée)
\documentclass[city=Montpellier,year=2015]{jres}

    % si pdflatex est utilisé, spécifier le codage du fichier source
    \usepackage[utf8x]{inputenc}

    \title{Titre de l'article JRES 2015}

    % Utiliser la commande \auteur pour definir la liste des auteurs.
    % Le 3e élément (adresse) peut être vide

    \auteur{Prénom Nom de l'auteur 1}
	{Laboratoire ou service...}{%
	    1, rue de l'adresse\\
	    34\,000 Montpellier}

    \auteur{Prénom Nom de l'auteur 2}
	{Laboratoire ou service}{%
	    2, rue du second auteur\\
	    34\,000 Montpellier}

    % Pas besoin de \date{}.

\begin{document}

% produit le titre selon les infos ci-dessus
\maketitle

% Résumé
\begin{abstract}
Ce document résume les instructions aux auteurs de la conférence
\href{http://www.jres.org}{JRES~2015}.

Le résumé doit faire de quinze à vingt lignes. Il est écrit en italique,
utilisant une police avec empattement en taille 10 points.
Le titre « Résumé » (ainsi que le titre « Mots-clefs » ci-dessous)
sont en 14 points, gras, sans empattement.
\end{abstract}

\motscles{IPv6, PKI, CMS...}

\section{Introduction}

Voici les instructions de présentation pour la rédaction d'un article
pour les JRES en utilisant \LaTeX{}.

Il est recommandé d'utiliser un moteur \LaTeX{} produisant directement
du PDF (\texttt{pdflatex}, \texttt{xelatex} ou \texttt{luatex}) et il est
impératif d'utiliser le format « Portable Document Format » (.pdf) pour
soumettre un article au comité de programme.


\section{Présentation générale}

\subsection{Structuration et polices}

La police de base est de type \emph{Times} pour le corps du texte, et
\textsf{Helvetica} pour les titres.
Le texte  normal est en 10 points.

La police pour les extraits de fichiers de configuration ou de code
source est \texttt{Courier} en 10 points.

L'environnement pour les extraits de fichiers de configuration ou
de code source  est « lstlisting ». Voici un exemple :
\vspace{2ex}
\begin{lstlisting}
10.0.0.1   serveur-1
127.0.0.1  localhost.localdomain localhost
\end{lstlisting}

Les titres de section sont en 14 points gras sans empattement
(Helvetica ou Arial), avec numérotation arabe.

Les titres de sous-section sont en 12 points gras sans empattement,
avec numérotation arabe (avec rappel du numéro de section et
séparation par un point).

Les titres de sous-sous-section sont en 10 points gras sans empattement,
avec numérotation arabe (avec rappel du numéro de sous-section et
séparation par un point).

Les titres de niveau inférieurs sont à éviter, pour des raisons de
lisibilité.
Les titres sont en couleur bleue sombre (valeurs RGB : 0.,0.2,0.375).  

Les notes de bas de page sont en 8 points\footnote{Comme ceci}.

\subsection{Typographie}

Les auteurs s'attacheront, dans la mesure du possible, à respecter les
règles typographiques françaises.

En particulier, ils veilleront à utiliser des espaces insécables avant
les ponctuations hautes (;:!?) et des guillemets français (« »).

Il est suggéré aux auteurs de lire les leçons de
typographie \cite{andre1990} de Jacques André.

\subsection{Bibliographie}

La bibliographie sera faite avec des numéros et dans l'ordre
d'apparition dans le texte en utilisant le style de bibliographie
\emph{jres} (voir fichier « article.bib »).

\begin{enumerate}

    \item Pour les articles d'une revue, on indiquera obligatoirement les
      auteurs, le titre de l'article, la revue, le numéro, le volume,
      l'année et le mois. Voir \cite{exemple1}

    \item Pour les articles dans les actes d'un congrès, on mentionnera
      obligatoirement les auteurs, le titre de l'article, le titre des
      actes, l'année. Voir \cite{exemple2} (le mois et le lieu sont
      facultatifs mais conseillés)

    \item Pour les livres, on citera obligatoirement le ou les auteurs, le
      titre, l'éditeur et l'année. Voir \cite{exemple3} (il est conseillé de
      rajouter le lieu d'édition)

\end{enumerate}

\subsection{Adresses électroniques}

Pour diminuer les risques de moissonnage, aucune adresse mail ne
figurera dans l'article. 

Pour les références à des pages web (sauf s'il s'agit d'une référence
auquel cas elles doivent être placées dans la bibliographie), la macro
\texttt{\char`\\url} sera utilisée, par exemple comme ceci :
\url{http://www.jres.org/}

\section{Mise en page}

\subsection{Puces et listes à numérotation}

Que ce soit pour les puces ou les listes à 
numérotation, il est conseillé de n'utiliser que deux niveaux, en
suivant les exemples ci-dessous :

\begin{itemize}
\item exemple de niveau 1 pour les puces;
\begin{itemize}
\item exemple de niveau 2 pour les puces;
\end{itemize}
\item deuxième exemple de niveau 1 pour les puces;
\end{itemize}
\begin{enumerate}
\item exemple de niveau 1 pour les listes numérotées;
\begin{enumerate}
\item exemple de niveau 2 pour les listes numérotées;
\item exemple de niveau 2 pour les listes numérotées second item;
\end{enumerate}
\item deuxième exemple de niveau 1 pour les listes numérotées.
\end{enumerate}

\subsection{Taille des articles}

Pour les présentations courtes, l'article doit comprendre entre 5000
et 25000 signes ; pour les présentations longues, la taille de l'article
doit être comprise entre 15000 et 25000 signes. Dans le cas des posters,
la taille doit être comprise entre 2500 et 25000 signes, le poster étant
lui même inséré avec cet article.

\subsection{Marges}

Les marges sont les suivantes :

\begin{itemize}
    \item en haut : 3\ cm ;
    \item à gauche et à droite : 3\ cm ;
    \item en bas : 3\ cm.
\end{itemize}

Les pages doivent être numérotées. Le pied de page du modèle doit être
maintenu.

\subsection{Figures}

Les figures sont de préférence insérées à proximité du texte qui y
fait référence. Elles sont numérotées et suivies d'une légende
explicite, en italique, non gras. Le texte fait référence à la
figure~\ref{fig-exemple}.

Le paquet \texttt{graphicx} est utilisé pour inclure des images.

\begin{figure}[hbtf]
    \centerline{\includegraphics[height=4cm]{figure}}
    \caption{Un exemple de figure}
    \label{fig-exemple}
\end{figure}

En cas d'insertion de figure contenant des couleurs, il est souhaitable
d'en assurer la lisibilité sur une imprimante noir et blanc.

\section*{Annexe}

Les auteurs ajoutent, si nécessaire, une annexe avec une section non
numérotée.

% Bibiliographie
\nocite{*}
\bibliography{article}

\end{document}
