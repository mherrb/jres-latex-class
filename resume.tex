%% Copyright 2012 CNRS LAAS 
%
% This work may be distributed and/or modified under the
% conditions of the LaTeX Project Public License, either version 1.3
% of this license or (at your option) any later version.
% The latest version of this license is in
%   http://www.latex-project.org/lppl.txt
% and version 1.3 or later is part of all distributions of LaTeX
% version 2005/12/01 or later.
%
% This work has the LPPL maintenance status `maintained'.
% 
% The Current Maintainer of this work is Matthieu Herrb
%
% This work consists of the files jres-resume.cls and resume.tex
\documentclass{jres-resume}

\title{Titre, Modèle de résumé JRES 2011 (centré, 14 points gras)}

\auteur{Prénom Nom de l'auteur 1 (Arial Narrow, 12  points)}%
{Laboratoire ou service … (Arial Narrow, 10 points)}

\auteur{Prénom Nom de l'auteur 2 (Arial Narrow, 12 points)}%
{Laboratoire ou service … (Arial Narrow,, 10 points)}

\begin{document}

\maketitle

% Le titre « Mots clefs » est en Arial Narrow, 12 points, gras.
\motscles{IPv6, PKI, CMS … (Arial Narrow, 12 points).}

\begin{abstract}
Le résumé doit faire de quinze à vingt lignes. Il est écrit en Arial
Narrow, 12 points. Le titre «~Résumé~» est en Arial Narrow,
12 points, gras.

L'ensemble du document doit tenir sur \textbf{une seule page au format A5}.

Ne pas faire figurer d'adresse mail dans votre résumé.

Veuillez à utiliser les styles utilisés dans ce modèle.

\end{abstract}
\end{document}
